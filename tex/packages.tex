\ifdefined\layoutletter
  \documentclass[fontsize=11pt, paper=a4,
    enlargefirstpage=on, pagenumber=headright,
    headsepline=on, parskip=half, foldmarks=off]{scrlttr2}
\else
  \ifdefined\layouttwelvept
    \documentclass[a4paper, 12pt, bibtotoc]{scrartcl}
  \else
    \ifdefined\layoutsmall
      \documentclass[twocolumn, paper=a4, fontsize=9pt]{scrartcl}
    \else
      \ifdefined\layoutnice
        \documentclass[paper=a4, fontsize=12pt, bibtotoc]{scrartcl}
      \else
        \documentclass[paper=a4, fontsize=11pt, bibtotoc]{scrartcl}
      \fi
    \fi
  \fi
\fi

\usepackage[ngerman]{babel}
\usepackage{graphicx}
\usepackage[utf8]{inputenc}
\usepackage[T1]{fontenc}
\usepackage{setspace}
%\usepackage{ellipsis}
\usepackage[fleqn]{amsmath}
\usepackage{amssymb}
\usepackage{mathdots}
\usepackage{rotating} % for \Lor
\usepackage{microtype}
\usepackage[pdftex]{hyperref}
\usepackage{paralist}
%\usepackage[stable]{footmisc} % for footnotes in \section{}s

\ifdefined\layoutmath
  \newcommand{\layoutcm}{}
  \newcommand{\layoutstd}{}
\fi

\ifdefined\layoutcm % computer modern
\else
  \usepackage{mathptmx}
  \usepackage{eucal} % more pretty power set in times font
\fi

\renewenvironment{quote}{\list{}{\leftmargin=2em\rightmargin=2em\small}\item[]}{\endlist}
\expandafter\def\expandafter\quote\expandafter{\quote\vspace{-1.1em}\singlespacing}

\newenvironment{defs}{\list{}{\leftmargin=1em\rightmargin=0em}\item[]\begin{compactitem}}{\end{compactitem}\endlist}

\ifdefined\layoutletter
  \setkomafont{fromname}{\sffamily \LARGE}
  \setkomafont{fromaddress}{\sffamily}%% statt \small
  \setkomafont{pagenumber}{\sffamily}
  \setkomafont{subject}{\mdseries\bfseries}
  \setkomafont{backaddress}{\mdseries}
  \usepackage[bottom=2cm]{geometry}
\else
  \ifdefined\layoutthesenpapier
    \usepackage[top=2cm, bottom=3cm, left=2.5cm, right=2.5cm]{geometry}
  \else
    \ifdefined\layoutsmall
      \usepackage[top=0.6cm, bottom=0.6cm, left=0.6cm, right=0.8cm]{geometry}
    \else
      \ifdefined\layoutnice
        \usepackage[top=3cm, bottom=3.3cm, left=3.5cm, right=4cm]{geometry}
      \else
        \usepackage[top=2cm, bottom=3cm, left=3cm, right=3cm]{geometry}
      \fi
    \fi
  \fi

  \usepackage[sort&compress]{natbib}
  \bibpunct{(}{)}{;}{a}{}{,}

  \ifdefined\layoutstd
    % we don't want fancy sections
  \else
    \usepackage{titlesec}
    \ifdefined\layoutnice
      %\titleformat{\section}{\singlespacing\normalfont\large\bfseries}{\thesection.}{0.5em}{}
      \titleformat{\section}{\singlespacing\normalsize\itshape}{}{0em}{}
      \titleformat{\subsection}{\singlespacing\normalsize\bfseries}{\thesubsection}{0.5em}{}
    \else
      %\renewcommand{\thesection}{\Roman{section}}
      \titleformat{\section}{\singlespacing\normalfont\Large}{\thesection.}{0.5em}{}
      %\titleformat{\subsection}{\singlespacing\normalfont\bfseries\large}{\thesubsection}{0.5em}{}
      %\titleformat{\subsection}{\singlespacing\normalsize\itshape}{}{0em}{}
      \titleformat{\subsection}{\singlespacing\normalsize\itshape}{\thesubsection}{0.5em}{}
    \fi
  \fi

  \usepackage{tocloft}
  \renewcommand{\cfttoctitlefont}{\hfill\normalfont\bfseries\large}
  \renewcommand{\cftaftertoctitle}{\hfill}
  \renewcommand{\cftsecleader}{\hfill}
  \renewcommand{\cftsecpagefont}{\normalfont}
  \renewcommand{\cftsubsecleader}{\hfill}
  \renewcommand{\cftsecfont}{\normalfont}
  \renewcommand{\cftsubsecfont}{\itshape}
  \tocloftpagestyle{empty}
\fi

\ifdefined\layoutlogik
  %\setlength{\plpartopsep}{12pt}
  \setlength{\pltopsep}{0pt}
  \setlength{\plitemsep}{4pt}
  %\defaultleftmargin{21pt}{24pt}{6pt}{6pt}
  % \pagestyle{empty}
\fi

\ifdefined\layoutcompact
  %\setlength{\plpartopsep}{12pt}
  \setlength{\pltopsep}{0pt}
  \setlength{\plitemsep}{6pt}
  %\defaultleftmargin{21pt}{24pt}{6pt}{6pt}
  \pagestyle{empty}
\fi

\ifdefined\layoutthesenpapier
  \setlength{\parindent}{0pt}
  \setlength{\parskip}{6pt}
  \setlength{\plpartopsep}{12pt}
  \setlength{\pltopsep}{0pt}
  \setlength{\plitemsep}{6pt}
  \defaultleftmargin{21pt}{24pt}{6pt}{6pt}
  \pagestyle{empty}
\fi

\renewcommand{\ldots}{...}

\newcommand{\ZZ}{\mathbb{Z}}
\newcommand{\RR}{\mathbb{R}}
\newcommand{\NN}{\mathbb{N}}
\newcommand{\QQ}{\mathbb{Q}}
\newcommand{\CC}{\mathbb{C}}
\newcommand{\FF}{\mathbb{F}}
\newcommand{\KK}{\mathbb{K}}
\newcommand{\PP}{\mathcal{P}}
\newcommand{\ggT}{\text{ggT}}
\DeclareMathOperator{\rang}{rang}
\DeclareMathOperator{\rg}{rg}
\DeclareMathOperator{\Rang}{Rang}
\DeclareMathOperator{\Los}{Lös}
\DeclareMathOperator{\Ker}{Ker}
\DeclareMathOperator{\Bild}{Bild}
\DeclareMathOperator{\sgn}{sgn}
\DeclareMathOperator{\Hom}{Hom}
\DeclareMathOperator{\Id}{Id}
\renewcommand{\Re}{\operatorname{Re}}
\renewcommand{\Im}{\operatorname{Im}}
\DeclareMathOperator{\ord}{ord}
\renewcommand{\phi}{\varphi}
\newcommand{\lam}{\lambda}
\newcommand{\eps}{\varepsilon}
\newcommand{\simto}{\xrightarrow{\sim}}
\newcommand{\up}{\uparrow}
\newcommand{\down}{\downarrow}
\renewcommand{\span}{\text{span}} % http://tex.stackexchange.com/q/33264
\renewcommand{\char}{\text{char}}
\newcommand{\Abb}{\text{Abb}}
\newcommand{\sse}{\subseteq}
\newcommand{\ssne}{\subsetneq}
\newcommand{\Produkt}{Produkt} % TODO
\newcommand{\qed}{ \hfill\square} % TODO: Space nötig?
\newcommand{\limn}{\displaystyle \lim_{n \to \infty}}
\newcommand{\limx}{\displaystyle \lim_{x \to \infty}}
\newcommand{\smatrix}[1]{\begin{smallmatrix}#1\end{smallmatrix}}
\newcommand{\mat}[1]{\left(\begin{smallmatrix}#1\end{smallmatrix}\right)}
\newcommand{\vm}[1]{\left|\begin{smallmatrix}#1\end{smallmatrix}\right|}
\providecommand{\abs}[1]{\lvert#1\rvert}
\providecommand{\norm}[1]{\lVert#1\rVert}
\newcommand{\ora}[1]{\overrightarrow{#1}}

\DeclareMathOperator{\Wert}{Wert}
\newcommand{\SKLAL}{\mathcal{SK}_{\mathcal{L}_\text{AL}}}
\newcommand{\FmLAL}{\mathcal{F}m_{\mathcal{L}_\text{AL}}}

% meta language
\newcommand{\Forall}{\raisebox{0.15em}{\scalebox{0.72}{$\backslash$}}\hspace{-0.18em}\forall}
\newcommand{\Exists}{\exists\hspace{-0.45em}\exists}
\newcommand{\Neg}{\neg\hspace{-0.5em}\neg}
\newcommand{\Lor}{\;\raisebox{-0.12em}{\begin{turn}{90}$\eqslantless$\end{turn}}\;}
\newcommand{\Land}{\;\raisebox{-0.12em}{\begin{turn}{90}$\eqslantgtr$\end{turn}}\;}

\newcommand{\concat}{\,\raisebox{0.45em}{\scalebox{0.7}{$\smallfrown$}}\,}
%\newcommand{\concat}{\,\raisebox{0.3em}{$\smallfrown$}\,}

\DeclareMathSymbol{\mlq}{\mathord}{operators}{``}
\DeclareMathSymbol{\mrq}{\mathord}{operators}{`'}

